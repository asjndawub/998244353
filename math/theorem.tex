\begin{itemize}
\item Lucas’ Theorem :\\
  For $n, m \in \mathbb{Z}^{*}$ and prime $P$,
  $C(m,n) \mod P$
  %= C(\frac{m}{M},n/M) * C(m\%M,n\%M) mod P
	$= \Pi ( C(m_i,n_i) )$
	where $m_i$ is the $i$-th digit of $m$ in base $P$.
\item Stirling Numbers(permutation $|P|=n$ with $k$ cycles): \\
  $S(n,k) = \text{coefficient of }x^k \text{ in } \Pi_{i=0}^{n-1} (x+i)$
\item Stirling Numbers(Partition $n$ elements into $k$ non-empty set): \\
  $S(n,k) = \frac{1}{k!} \sum\limits_{j=0}^k (-1)^{k-j} {k \choose j} j^n$
\item Pick’s Theorem : $A = i + b/2 - 1$
\item Kirchhoff's theorem :\\
  $A_{ii} = deg(i), A_{ij} = (i,j) \in E\ ? -1 : 0$,
  Deleting any one row, one column, and cal the det(A)
\item Burnside Lemma:
  $|X/G|=\frac{1}{|G|}\sum\limits_{g \in G} {|X^g|}$
\item Polya theorem:
  $|Y^x/G|=\frac{1}{|G|}\sum\limits_{g \in G} {m^{c(g)}}$\\
  $m = |Y|$ : num of colors, c(g) : num of cycle
\item Anti SG (the person who has no strategy wins) :\\
  first player wins iff either\\
  1. SG value of ALL subgame $\le$ $1$ and SG value of the game $=$ $0$\\
  2. SG value of some subgame $>$ $1$ and SG value of the game $\neq$ $0$
\item Möbius inversion formula :\\
  $g(n) = \sum\limits_{d|n}f(d)$ for every integer $n\ge 1$ , then\\
  $f(n) = \sum\limits_{d|n}\mu(d)g(\frac{n}{d}) = \sum\limits_{d|n}\mu(\frac{n}{d})g(d)$ for every integer $n\ge 1$\\
  %$\sum\limits_{d|n}\mu(d)=\epsilon(n)=[n=1]$ , 
  %$\sum\limits_{d|n}\phi(d)=n\Leftrightarrow \phi(n)=\sum\limits_{d|n}\mu(d)(\frac{n}{d})$\\
  Dirichlet convolution : $f*g=g*f=\sum\limits_{d|n}f(d)g(\frac{n}{d})=\sum\limits_{d|n}f(\frac{n}{d})g(d)$\\
  $g=f*1\Leftrightarrow f=g*\mu$, $\epsilon=\mu*1$, $Id=\phi*1$, $d=1*1$, $\sigma=Id*1=\phi*d$,\\
  $\sigma_k=Id_k*1$ where $\epsilon(n)=[n=1]$, $1(n)=1$, $Id(n)=n$, $Id_k(n)=n^k$,\\
  $d(n)=\#(divisor)$, $\sigma(n)=\sum divisor$, $\sigma_k(n)=\sum divisor^k$
\item Find a Primitive Root of $n$:\\
  $n$ has primitive roots iff $n=2,4,p^k,2p^k$ where $p$ is an odd prime.\\
  1. Find $\phi(n)$ and all prime factors of $\phi(n)$, says $P=\{p_1,...,p_m\}$\\
  2. $\forall g\in[2,n)$, if $g^{\frac{\phi(n)}{p_i}}\ne 1,\forall p_i\in P$, then $g$ is a primitive root.\\
  3. Since the smallest one isn't too big, the algorithm runs fast.\\
  4. $n$ has exactly $\phi(\phi(n))$ primitive roots.
\end{itemize}
